\documentclass[]{article}
\usepackage{lmodern}
\usepackage{amssymb,amsmath}
\usepackage{ifxetex,ifluatex}
\usepackage{fixltx2e} % provides \textsubscript
\ifnum 0\ifxetex 1\fi\ifluatex 1\fi=0 % if pdftex
  \usepackage[T1]{fontenc}
  \usepackage[utf8]{inputenc}
\else % if luatex or xelatex
  \ifxetex
    \usepackage{mathspec}
  \else
    \usepackage{fontspec}
  \fi
  \defaultfontfeatures{Ligatures=TeX,Scale=MatchLowercase}
\fi
% use upquote if available, for straight quotes in verbatim environments
\IfFileExists{upquote.sty}{\usepackage{upquote}}{}
% use microtype if available
\IfFileExists{microtype.sty}{%
\usepackage{microtype}
\UseMicrotypeSet[protrusion]{basicmath} % disable protrusion for tt fonts
}{}
\usepackage[margin=1in]{geometry}
\usepackage{hyperref}
\hypersetup{unicode=true,
            pdftitle={Neural Networks},
            pdfauthor={Shantam Gupta},
            pdfborder={0 0 0},
            breaklinks=true}
\urlstyle{same}  % don't use monospace font for urls
\usepackage{color}
\usepackage{fancyvrb}
\newcommand{\VerbBar}{|}
\newcommand{\VERB}{\Verb[commandchars=\\\{\}]}
\DefineVerbatimEnvironment{Highlighting}{Verbatim}{commandchars=\\\{\}}
% Add ',fontsize=\small' for more characters per line
\usepackage{framed}
\definecolor{shadecolor}{RGB}{248,248,248}
\newenvironment{Shaded}{\begin{snugshade}}{\end{snugshade}}
\newcommand{\KeywordTok}[1]{\textcolor[rgb]{0.13,0.29,0.53}{\textbf{#1}}}
\newcommand{\DataTypeTok}[1]{\textcolor[rgb]{0.13,0.29,0.53}{#1}}
\newcommand{\DecValTok}[1]{\textcolor[rgb]{0.00,0.00,0.81}{#1}}
\newcommand{\BaseNTok}[1]{\textcolor[rgb]{0.00,0.00,0.81}{#1}}
\newcommand{\FloatTok}[1]{\textcolor[rgb]{0.00,0.00,0.81}{#1}}
\newcommand{\ConstantTok}[1]{\textcolor[rgb]{0.00,0.00,0.00}{#1}}
\newcommand{\CharTok}[1]{\textcolor[rgb]{0.31,0.60,0.02}{#1}}
\newcommand{\SpecialCharTok}[1]{\textcolor[rgb]{0.00,0.00,0.00}{#1}}
\newcommand{\StringTok}[1]{\textcolor[rgb]{0.31,0.60,0.02}{#1}}
\newcommand{\VerbatimStringTok}[1]{\textcolor[rgb]{0.31,0.60,0.02}{#1}}
\newcommand{\SpecialStringTok}[1]{\textcolor[rgb]{0.31,0.60,0.02}{#1}}
\newcommand{\ImportTok}[1]{#1}
\newcommand{\CommentTok}[1]{\textcolor[rgb]{0.56,0.35,0.01}{\textit{#1}}}
\newcommand{\DocumentationTok}[1]{\textcolor[rgb]{0.56,0.35,0.01}{\textbf{\textit{#1}}}}
\newcommand{\AnnotationTok}[1]{\textcolor[rgb]{0.56,0.35,0.01}{\textbf{\textit{#1}}}}
\newcommand{\CommentVarTok}[1]{\textcolor[rgb]{0.56,0.35,0.01}{\textbf{\textit{#1}}}}
\newcommand{\OtherTok}[1]{\textcolor[rgb]{0.56,0.35,0.01}{#1}}
\newcommand{\FunctionTok}[1]{\textcolor[rgb]{0.00,0.00,0.00}{#1}}
\newcommand{\VariableTok}[1]{\textcolor[rgb]{0.00,0.00,0.00}{#1}}
\newcommand{\ControlFlowTok}[1]{\textcolor[rgb]{0.13,0.29,0.53}{\textbf{#1}}}
\newcommand{\OperatorTok}[1]{\textcolor[rgb]{0.81,0.36,0.00}{\textbf{#1}}}
\newcommand{\BuiltInTok}[1]{#1}
\newcommand{\ExtensionTok}[1]{#1}
\newcommand{\PreprocessorTok}[1]{\textcolor[rgb]{0.56,0.35,0.01}{\textit{#1}}}
\newcommand{\AttributeTok}[1]{\textcolor[rgb]{0.77,0.63,0.00}{#1}}
\newcommand{\RegionMarkerTok}[1]{#1}
\newcommand{\InformationTok}[1]{\textcolor[rgb]{0.56,0.35,0.01}{\textbf{\textit{#1}}}}
\newcommand{\WarningTok}[1]{\textcolor[rgb]{0.56,0.35,0.01}{\textbf{\textit{#1}}}}
\newcommand{\AlertTok}[1]{\textcolor[rgb]{0.94,0.16,0.16}{#1}}
\newcommand{\ErrorTok}[1]{\textcolor[rgb]{0.64,0.00,0.00}{\textbf{#1}}}
\newcommand{\NormalTok}[1]{#1}
\usepackage{graphicx,grffile}
\makeatletter
\def\maxwidth{\ifdim\Gin@nat@width>\linewidth\linewidth\else\Gin@nat@width\fi}
\def\maxheight{\ifdim\Gin@nat@height>\textheight\textheight\else\Gin@nat@height\fi}
\makeatother
% Scale images if necessary, so that they will not overflow the page
% margins by default, and it is still possible to overwrite the defaults
% using explicit options in \includegraphics[width, height, ...]{}
\setkeys{Gin}{width=\maxwidth,height=\maxheight,keepaspectratio}
\IfFileExists{parskip.sty}{%
\usepackage{parskip}
}{% else
\setlength{\parindent}{0pt}
\setlength{\parskip}{6pt plus 2pt minus 1pt}
}
\setlength{\emergencystretch}{3em}  % prevent overfull lines
\providecommand{\tightlist}{%
  \setlength{\itemsep}{0pt}\setlength{\parskip}{0pt}}
\setcounter{secnumdepth}{5}
% Redefines (sub)paragraphs to behave more like sections
\ifx\paragraph\undefined\else
\let\oldparagraph\paragraph
\renewcommand{\paragraph}[1]{\oldparagraph{#1}\mbox{}}
\fi
\ifx\subparagraph\undefined\else
\let\oldsubparagraph\subparagraph
\renewcommand{\subparagraph}[1]{\oldsubparagraph{#1}\mbox{}}
\fi

%%% Use protect on footnotes to avoid problems with footnotes in titles
\let\rmarkdownfootnote\footnote%
\def\footnote{\protect\rmarkdownfootnote}

%%% Change title format to be more compact
\usepackage{titling}

% Create subtitle command for use in maketitle
\newcommand{\subtitle}[1]{
  \posttitle{
    \begin{center}\large#1\end{center}
    }
}

\setlength{\droptitle}{-2em}
  \title{Neural Networks}
  \pretitle{\vspace{\droptitle}\centering\huge}
  \posttitle{\par}
  \author{Shantam Gupta}
  \preauthor{\centering\large\emph}
  \postauthor{\par}
  \predate{\centering\large\emph}
  \postdate{\par}
  \date{May 13, 2018}


\begin{document}
\maketitle

{
\setcounter{tocdepth}{2}
\tableofcontents
}
\section{Installing the Package}\label{installing-the-package}

\section{Load the Data}\label{load-the-data}

\section{Preprocess the data: Normalize the
data}\label{preprocess-the-data-normalize-the-data}

\begin{Shaded}
\begin{Highlighting}[]
\NormalTok{new_data <-}\StringTok{ }\KeywordTok{rbind}\NormalTok{(S0,Data)}
\NormalTok{maxs <-}\StringTok{ }\KeywordTok{apply}\NormalTok{(new_data }\OperatorTok\StringTok{ }\KeywordTok{select}\NormalTok{(}\OperatorTok{-}\KeywordTok{c}\NormalTok{(idfile,RESPONSE)), }\DecValTok{2}\NormalTok{, max) }
\NormalTok{mins <-}\StringTok{ }\KeywordTok{apply}\NormalTok{(new_data }\OperatorTok\StringTok{ }\KeywordTok{select}\NormalTok{(}\OperatorTok{-}\KeywordTok{c}\NormalTok{(idfile,RESPONSE)), }\DecValTok{2}\NormalTok{, min)}

\NormalTok{scaled_data <-}\StringTok{ }\KeywordTok{as.data.frame}\NormalTok{(}\KeywordTok{scale}\NormalTok{(new_data }\OperatorTok\StringTok{ }\KeywordTok{select}\NormalTok{(}\OperatorTok{-}\KeywordTok{c}\NormalTok{(idfile,RESPONSE)), }\DataTypeTok{center =}\NormalTok{ mins, }\DataTypeTok{scale =}\NormalTok{ maxs }\OperatorTok{-}\StringTok{ }\NormalTok{mins))}
\CommentTok{#scaled_data$RESPONSE <- ifelse(new_data$RESPONSE =="GO",1,0)}
\NormalTok{scaled_data}\OperatorTok{$}\NormalTok{RESPONSE <-}\StringTok{ }\KeywordTok{as.factor}\NormalTok{(new_data}\OperatorTok{$}\NormalTok{RESPONSE)}
\CommentTok{#scaled_data$RESPONSE <- as.factor(scaled_data$RESPONSE)}
\NormalTok{scaled_data}\OperatorTok{$}\NormalTok{idfile <-}\StringTok{ }\NormalTok{new_data}\OperatorTok{$}\NormalTok{idfile}

\CommentTok{#select random ind for train and test }
\KeywordTok{set.seed}\NormalTok{(}\DecValTok{123}\NormalTok{)}

\NormalTok{## 75% of the sample size}
\NormalTok{smp_size <-}\StringTok{ }\KeywordTok{floor}\NormalTok{(}\FloatTok{0.75} \OperatorTok{*}\StringTok{ }\KeywordTok{nrow}\NormalTok{(scaled_data))}

\NormalTok{## set the seed to make your partition reproducible}
\KeywordTok{set.seed}\NormalTok{(}\DecValTok{123}\NormalTok{)}
\NormalTok{train_ind <-}\StringTok{ }\KeywordTok{sample}\NormalTok{(}\KeywordTok{seq_len}\NormalTok{(}\KeywordTok{nrow}\NormalTok{(scaled_data)), }\DataTypeTok{size =}\NormalTok{ smp_size)}


\NormalTok{train <-}\StringTok{ }\NormalTok{scaled_data[train_ind,]}
\NormalTok{test <-}\StringTok{ }\NormalTok{scaled_data[}\OperatorTok{-}\NormalTok{train_ind,]}
\end{Highlighting}
\end{Shaded}

\section{Building Neural Network}\label{building-neural-network}

\begin{Shaded}
\begin{Highlighting}[]
\KeywordTok{library}\NormalTok{(h2o)}
\end{Highlighting}
\end{Shaded}

\begin{verbatim}
## 
## ----------------------------------------------------------------------
## 
## Your next step is to start H2O:
##     > h2o.init()
## 
## For H2O package documentation, ask for help:
##     > ??h2o
## 
## After starting H2O, you can use the Web UI at http://localhost:54321
## For more information visit http://docs.h2o.ai
## 
## ----------------------------------------------------------------------
\end{verbatim}

\begin{verbatim}
## 
## Attaching package: 'h2o'
\end{verbatim}

\begin{verbatim}
## The following objects are masked from 'package:stats':
## 
##     cor, sd, var
\end{verbatim}

\begin{verbatim}
## The following objects are masked from 'package:base':
## 
##     %*%, %in%, &&, ||, apply, as.factor, as.numeric, colnames,
##     colnames<-, ifelse, is.character, is.factor, is.numeric, log,
##     log10, log1p, log2, round, signif, trunc
\end{verbatim}

\begin{Shaded}
\begin{Highlighting}[]
\CommentTok{#generate same set of random numbers (for reproducibility)}
\KeywordTok{set.seed}\NormalTok{(}\DecValTok{121}\NormalTok{)}

\CommentTok{#launch h2o cluster}
\NormalTok{localH2O <-}\StringTok{ }\KeywordTok{h2o.init}\NormalTok{(}\DataTypeTok{nthreads =} \OperatorTok{-}\DecValTok{1}\NormalTok{)}


\CommentTok{#import r objects to h2o cloud}
\NormalTok{train_h2o <-}\StringTok{ }\KeywordTok{as.h2o}\NormalTok{(train)}
\NormalTok{test_h2o <-}\StringTok{ }\KeywordTok{as.h2o}\NormalTok{(test)}
\end{Highlighting}
\end{Shaded}

\begin{Shaded}
\begin{Highlighting}[]
\CommentTok{#build the mlp(multi layer perceptron) deep learning model using h2o}
\KeywordTok{set.seed}\NormalTok{(}\DecValTok{100}\NormalTok{)}

\NormalTok{dl_model <-}\StringTok{ }\KeywordTok{h2o.deeplearning}\NormalTok{(}
  \DataTypeTok{model_id=}\StringTok{"dl_model_first"}\NormalTok{, }
  \DataTypeTok{training_frame=}\NormalTok{train_h2o, }
  \DataTypeTok{validation_frame =}\NormalTok{ test_h2o,}
  \DataTypeTok{x=} \KeywordTok{colnames}\NormalTok{(train_h2o[,}\DecValTok{1}\OperatorTok{:}\DecValTok{48}\NormalTok{]),}
  \DataTypeTok{y=} \StringTok{"RESPONSE"}\NormalTok{,}
  \DataTypeTok{activation=}\StringTok{"Rectifier"}\NormalTok{,  }
  \DataTypeTok{hidden=}\KeywordTok{c}\NormalTok{(}\DecValTok{5}\NormalTok{,}\DecValTok{4}\NormalTok{), }
  \DataTypeTok{stopping_metric=}\StringTok{"mean_per_class_error"}\NormalTok{,}
  \DataTypeTok{stopping_tolerance=}\FloatTok{0.01}\NormalTok{,}
  \DataTypeTok{epochs=}\DecValTok{100}        
\NormalTok{)}
\end{Highlighting}
\end{Shaded}

\begin{Shaded}
\begin{Highlighting}[]
\KeywordTok{summary}\NormalTok{(dl_model)}
\end{Highlighting}
\end{Shaded}

\begin{verbatim}
## Model Details:
## ==============
## 
## H2OBinomialModel: deeplearning
## Model Key:  dl_model_first 
## Status of Neuron Layers: predicting RESPONSE, 2-class classification, bernoulli distribution, CrossEntropy loss, 279 weights/biases, 16.2 KB, 105,600 training samples, mini-batch size 1
##   layer units      type dropout       l1       l2 mean_rate rate_rms
## 1     1    48     Input  0.00 %                                     
## 2     2     5 Rectifier  0.00 % 0.000000 0.000000  0.005227 0.004851
## 3     3     4 Rectifier  0.00 % 0.000000 0.000000  0.004046 0.005787
## 4     4     2   Softmax         0.000000 0.000000  0.001208 0.000238
##   momentum mean_weight weight_rms mean_bias bias_rms
## 1                                                   
## 2 0.000000   -0.012467   0.246006  0.608905 0.121557
## 3 0.000000    0.132601   0.830154  1.048259 0.200680
## 4 0.000000   -0.559657   2.181219  0.005555 0.047890
## 
## H2OBinomialMetrics: deeplearning
## ** Reported on training data. **
## ** Metrics reported on full training frame **
## 
## MSE:  0.007385807
## RMSE:  0.08594072
## LogLoss:  0.03213156
## Mean Per-Class Error:  0.02012779
## AUC:  0.9974616
## Gini:  0.9949232
## 
## Confusion Matrix (vertical: actual; across: predicted) for F1-optimal threshold:
##        GO NOGO    Error     Rate
## GO     80    3 0.036145    =3/83
## NOGO    4  969 0.004111   =4/973
## Totals 84  972 0.006629  =7/1056
## 
## Maximum Metrics: Maximum metrics at their respective thresholds
##                         metric threshold    value idx
## 1                       max f1  0.543928 0.996401 315
## 2                       max f2  0.446713 0.996302 319
## 3                 max f0point5  0.711292 0.997519 309
## 4                 max accuracy  0.543928 0.993371 315
## 5                max precision  1.000000 1.000000   0
## 6                   max recall  0.002332 1.000000 399
## 7              max specificity  1.000000 1.000000   0
## 8             max absolute_mcc  0.543928 0.954505 315
## 9   max min_per_class_accuracy  0.790724 0.987952 306
## 10 max mean_per_class_accuracy  0.711292 0.989865 309
## 
## Gains/Lift Table: Extract with `h2o.gainsLift(<model>, <data>)` or `h2o.gainsLift(<model>, valid=<T/F>, xval=<T/F>)`
## H2OBinomialMetrics: deeplearning
## ** Reported on validation data. **
## ** Metrics reported on full validation frame **
## 
## MSE:  0.008362441
## RMSE:  0.09144638
## LogLoss:  0.02741105
## Mean Per-Class Error:  0.02083333
## AUC:  0.9997467
## Gini:  0.9994934
## 
## Confusion Matrix (vertical: actual; across: predicted) for F1-optimal threshold:
##        GO NOGO    Error    Rate
## GO     23    1 0.041667   =1/24
## NOGO    0  329 0.000000  =0/329
## Totals 23  330 0.002833  =1/353
## 
## Maximum Metrics: Maximum metrics at their respective thresholds
##                         metric threshold    value idx
## 1                       max f1  0.868842 0.998483 316
## 2                       max f2  0.868842 0.999392 316
## 3                 max f0point5  0.919001 0.998778 313
## 4                 max accuracy  0.868842 0.997167 316
## 5                max precision  1.000000 1.000000   0
## 6                   max recall  0.868842 1.000000 316
## 7              max specificity  1.000000 1.000000   0
## 8             max absolute_mcc  0.868842 0.977461 316
## 9   max min_per_class_accuracy  0.919001 0.993921 313
## 10 max mean_per_class_accuracy  0.919001 0.996960 313
## 
## Gains/Lift Table: Extract with `h2o.gainsLift(<model>, <data>)` or `h2o.gainsLift(<model>, valid=<T/F>, xval=<T/F>)`
## 
## 
## Scoring History: 
##             timestamp   duration training_speed    epochs iterations
## 1 2018-05-17 20:30:21  0.000 sec                  0.00000          0
## 2 2018-05-17 20:30:21  0.101 sec 224680 obs/sec  10.00000          1
## 3 2018-05-17 20:30:21  0.399 sec 324923 obs/sec 100.00000         10
##         samples training_rmse training_logloss training_auc training_lift
## 1      0.000000                                                          
## 2  10560.000000       0.18376          0.11958      0.96312       1.08530
## 3 105600.000000       0.08594          0.03213      0.99746       1.08530
##   training_classification_error validation_rmse validation_logloss
## 1                                                                 
## 2                       0.03883         0.16542            0.09604
## 3                       0.00663         0.09145            0.02741
##   validation_auc validation_lift validation_classification_error
## 1                                                               
## 2        0.97480         1.07295                         0.02550
## 3        0.99975         1.07295                         0.00283
## 
## Variable Importances: (Extract with `h2o.varimp`) 
## =================================================
## 
## Variable Importances: 
##                variable relative_importance scaled_importance percentage
## 1  TotalArea.LVNELTEFAK            1.000000          1.000000   0.044828
## 2  MassAccu.HLVDEPQNLIK            0.950939          0.950939   0.042629
## 3 MassAccu.SLHTLFGDELCK            0.936479          0.936479   0.041980
## 4      FWHM.HLVDEPQNLIK            0.790054          0.790054   0.035417
## 5 Charge.VPQVSTPTLVEVSR            0.713035          0.713035   0.031964
## 
## ---
##               variable relative_importance scaled_importance percentage
## 43 MassAccu.LVNELTEFAK            0.300878          0.300878   0.013488
## 44   MassAccu.NECFLSHK            0.275860          0.275860   0.012366
## 45   MZ.ECCHGDLLECADDR            0.268748          0.268748   0.012047
## 46   MZ.VPQVSTPTLVEVSR            0.238041          0.238041   0.010671
## 47   Charge.EACFAVEGPK            0.233442          0.233442   0.010465
## 48   FWHM.YICDNQDTISSK            0.225689          0.225689   0.010117
\end{verbatim}

The accuracy is 98.57\% . The net could be optmizied further to improve
the accuracy

\subsubsection{Tuning the ANN}\label{tuning-the-ann}

The simplest hyperparameter search method is a brute-force scan of the
full Cartesian product of all combinations specified by a grid search.
There are a lot of paramters to tune and due to limited computational
capabilities we shall try to tune only some of them.

\begin{Shaded}
\begin{Highlighting}[]
\CommentTok{#hyperparamters to tune }
\NormalTok{hyper_params <-}\StringTok{ }\KeywordTok{list}\NormalTok{(}
  \DataTypeTok{hidden=}\KeywordTok{list}\NormalTok{(}\KeywordTok{c}\NormalTok{(}\DecValTok{32}\NormalTok{,}\DecValTok{32}\NormalTok{,}\DecValTok{32}\NormalTok{),}\KeywordTok{c}\NormalTok{(}\DecValTok{50}\NormalTok{,}\DecValTok{200}\NormalTok{,}\DecValTok{50}\NormalTok{)),  }\CommentTok{# different architectures of hidden layer}
  \DataTypeTok{input_dropout_ratio=}\KeywordTok{c}\NormalTok{(}\DecValTok{0}\NormalTok{,}\FloatTok{0.05}\NormalTok{),      }\CommentTok{# values for drop out}
  \DataTypeTok{rate=}\KeywordTok{c}\NormalTok{(}\FloatTok{0.01}\NormalTok{,}\FloatTok{0.02}\NormalTok{),                  }\CommentTok{# the learning rae}
  \DataTypeTok{activation =} \KeywordTok{c}\NormalTok{(}\StringTok{"Rectifier"}\NormalTok{)   }\CommentTok{# activation functions}
\NormalTok{)}

\CommentTok{#grid search}

\NormalTok{grid <-}\StringTok{ }\KeywordTok{h2o.grid}\NormalTok{(}
  \DataTypeTok{algorithm=}\StringTok{"deeplearning"}\NormalTok{,}
  \DataTypeTok{grid_id=}\StringTok{"dl_grid"}\NormalTok{,}
  \DataTypeTok{model_id=}\StringTok{"dl_model_first"}\NormalTok{, }
  \DataTypeTok{training_frame=}\NormalTok{train_h2o, }
  \DataTypeTok{x=} \KeywordTok{colnames}\NormalTok{(train_h2o[,}\DecValTok{1}\OperatorTok{:}\DecValTok{12}\NormalTok{]),}
  \DataTypeTok{y=} \StringTok{"label"}\NormalTok{,}
  \DataTypeTok{stopping_metric=}\StringTok{"mean_per_class_error"}\NormalTok{,}
  \DataTypeTok{hyper_params =}\NormalTok{ hyper_params,}
  \DataTypeTok{epochs=}\DecValTok{1000}\NormalTok{,   }
  \DataTypeTok{stopping_tolerance=}\FloatTok{0.01}\NormalTok{,}
  \DataTypeTok{variable_importances=}\NormalTok{T    }
\NormalTok{)}
\end{Highlighting}
\end{Shaded}

\begin{Shaded}
\begin{Highlighting}[]
\CommentTok{# sort the model in the grid in decreasing order of error}
\NormalTok{grid <-}\StringTok{ }\KeywordTok{h2o.getGrid}\NormalTok{(}\StringTok{"dl_grid"}\NormalTok{, }\DataTypeTok{sort_by =} \StringTok{"err"}\NormalTok{, }\DataTypeTok{decreasing =} \OtherTok{FALSE}\NormalTok{)}
\NormalTok{grid}

\CommentTok{# best model and its full set of parameters}
\NormalTok{grid}\OperatorTok{@}\NormalTok{summary_table[}\DecValTok{1}\NormalTok{, ]}
\NormalTok{best_dl_model <-}\StringTok{ }\KeywordTok{h2o.getModel}\NormalTok{(grid}\OperatorTok{@}\NormalTok{model_ids[[}\DecValTok{1}\NormalTok{]])}
\NormalTok{best_dl_model}

\KeywordTok{print}\NormalTok{(}\KeywordTok{h2o.performance}\NormalTok{(best_dl_model))}

\CommentTok{# storing the confusion matrix}
\NormalTok{best_dl_confusion <-}\StringTok{ }\KeywordTok{as.data.frame}\NormalTok{(}\KeywordTok{h2o.confusionMatrix}\NormalTok{(best_dl_model))}
\end{Highlighting}
\end{Shaded}

\subsubsection{Plotting the model}\label{plotting-the-model}

\begin{Shaded}
\begin{Highlighting}[]
\KeywordTok{plot}\NormalTok{(dl_model,}\DataTypeTok{timesteps =} \StringTok{"epochs"}\NormalTok{,}\DataTypeTok{metric =} \StringTok{"classification_error"}\NormalTok{)}
\end{Highlighting}
\end{Shaded}

\begin{verbatim}
## Warning in plot.window(...): "timesteps" is not a graphical parameter
\end{verbatim}

\begin{verbatim}
## Warning in plot.xy(xy, type, ...): "timesteps" is not a graphical parameter
\end{verbatim}

\begin{verbatim}
## Warning in title(...): "timesteps" is not a graphical parameter
\end{verbatim}

\begin{verbatim}
## Warning in plot.window(...): "timesteps" is not a graphical parameter
\end{verbatim}

\begin{verbatim}
## Warning in plot.xy(xy, type, ...): "timesteps" is not a graphical parameter
\end{verbatim}

\begin{verbatim}
## Warning in axis(side = side, at = at, labels = labels, ...): "timesteps" is
## not a graphical parameter

## Warning in axis(side = side, at = at, labels = labels, ...): "timesteps" is
## not a graphical parameter
\end{verbatim}

\begin{verbatim}
## Warning in box(...): "timesteps" is not a graphical parameter
\end{verbatim}

\begin{verbatim}
## Warning in title(...): "timesteps" is not a graphical parameter
\end{verbatim}

\includegraphics{NeuralNetworks_files/figure-latex/unnamed-chunk-10-1.pdf}

The training accuracy decreases with increase in epochs. However, this
might lead to overfitting on training data and poor fit on the test
data.

\subsubsection{Predictons on test data}\label{predictons-on-test-data}

\begin{Shaded}
\begin{Highlighting}[]
\NormalTok{dl_predict <-}\StringTok{ }\KeywordTok{as.data.frame}\NormalTok{(}\KeywordTok{h2o.predict}\NormalTok{(dl_model, test_h2o))}
\end{Highlighting}
\end{Shaded}

\begin{Shaded}
\begin{Highlighting}[]
\KeywordTok{h2o.varimp_plot}\NormalTok{(dl_model)}
\end{Highlighting}
\end{Shaded}

\includegraphics{NeuralNetworks_files/figure-latex/unnamed-chunk-12-1.pdf}


\end{document}
