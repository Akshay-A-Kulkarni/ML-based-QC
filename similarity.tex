\documentclass[]{article}
\usepackage{lmodern}
\usepackage{amssymb,amsmath}
\usepackage{ifxetex,ifluatex}
\usepackage{fixltx2e} % provides \textsubscript
\ifnum 0\ifxetex 1\fi\ifluatex 1\fi=0 % if pdftex
  \usepackage[T1]{fontenc}
  \usepackage[utf8]{inputenc}
\else % if luatex or xelatex
  \ifxetex
    \usepackage{mathspec}
  \else
    \usepackage{fontspec}
  \fi
  \defaultfontfeatures{Ligatures=TeX,Scale=MatchLowercase}
\fi
% use upquote if available, for straight quotes in verbatim environments
\IfFileExists{upquote.sty}{\usepackage{upquote}}{}
% use microtype if available
\IfFileExists{microtype.sty}{%
\usepackage{microtype}
\UseMicrotypeSet[protrusion]{basicmath} % disable protrusion for tt fonts
}{}
\usepackage[margin=1in]{geometry}
\usepackage{hyperref}
\hypersetup{unicode=true,
            pdftitle={Similarity},
            pdfauthor={Shantam Gupta},
            pdfborder={0 0 0},
            breaklinks=true}
\urlstyle{same}  % don't use monospace font for urls
\usepackage{color}
\usepackage{fancyvrb}
\newcommand{\VerbBar}{|}
\newcommand{\VERB}{\Verb[commandchars=\\\{\}]}
\DefineVerbatimEnvironment{Highlighting}{Verbatim}{commandchars=\\\{\}}
% Add ',fontsize=\small' for more characters per line
\usepackage{framed}
\definecolor{shadecolor}{RGB}{248,248,248}
\newenvironment{Shaded}{\begin{snugshade}}{\end{snugshade}}
\newcommand{\KeywordTok}[1]{\textcolor[rgb]{0.13,0.29,0.53}{\textbf{#1}}}
\newcommand{\DataTypeTok}[1]{\textcolor[rgb]{0.13,0.29,0.53}{#1}}
\newcommand{\DecValTok}[1]{\textcolor[rgb]{0.00,0.00,0.81}{#1}}
\newcommand{\BaseNTok}[1]{\textcolor[rgb]{0.00,0.00,0.81}{#1}}
\newcommand{\FloatTok}[1]{\textcolor[rgb]{0.00,0.00,0.81}{#1}}
\newcommand{\ConstantTok}[1]{\textcolor[rgb]{0.00,0.00,0.00}{#1}}
\newcommand{\CharTok}[1]{\textcolor[rgb]{0.31,0.60,0.02}{#1}}
\newcommand{\SpecialCharTok}[1]{\textcolor[rgb]{0.00,0.00,0.00}{#1}}
\newcommand{\StringTok}[1]{\textcolor[rgb]{0.31,0.60,0.02}{#1}}
\newcommand{\VerbatimStringTok}[1]{\textcolor[rgb]{0.31,0.60,0.02}{#1}}
\newcommand{\SpecialStringTok}[1]{\textcolor[rgb]{0.31,0.60,0.02}{#1}}
\newcommand{\ImportTok}[1]{#1}
\newcommand{\CommentTok}[1]{\textcolor[rgb]{0.56,0.35,0.01}{\textit{#1}}}
\newcommand{\DocumentationTok}[1]{\textcolor[rgb]{0.56,0.35,0.01}{\textbf{\textit{#1}}}}
\newcommand{\AnnotationTok}[1]{\textcolor[rgb]{0.56,0.35,0.01}{\textbf{\textit{#1}}}}
\newcommand{\CommentVarTok}[1]{\textcolor[rgb]{0.56,0.35,0.01}{\textbf{\textit{#1}}}}
\newcommand{\OtherTok}[1]{\textcolor[rgb]{0.56,0.35,0.01}{#1}}
\newcommand{\FunctionTok}[1]{\textcolor[rgb]{0.00,0.00,0.00}{#1}}
\newcommand{\VariableTok}[1]{\textcolor[rgb]{0.00,0.00,0.00}{#1}}
\newcommand{\ControlFlowTok}[1]{\textcolor[rgb]{0.13,0.29,0.53}{\textbf{#1}}}
\newcommand{\OperatorTok}[1]{\textcolor[rgb]{0.81,0.36,0.00}{\textbf{#1}}}
\newcommand{\BuiltInTok}[1]{#1}
\newcommand{\ExtensionTok}[1]{#1}
\newcommand{\PreprocessorTok}[1]{\textcolor[rgb]{0.56,0.35,0.01}{\textit{#1}}}
\newcommand{\AttributeTok}[1]{\textcolor[rgb]{0.77,0.63,0.00}{#1}}
\newcommand{\RegionMarkerTok}[1]{#1}
\newcommand{\InformationTok}[1]{\textcolor[rgb]{0.56,0.35,0.01}{\textbf{\textit{#1}}}}
\newcommand{\WarningTok}[1]{\textcolor[rgb]{0.56,0.35,0.01}{\textbf{\textit{#1}}}}
\newcommand{\AlertTok}[1]{\textcolor[rgb]{0.94,0.16,0.16}{#1}}
\newcommand{\ErrorTok}[1]{\textcolor[rgb]{0.64,0.00,0.00}{\textbf{#1}}}
\newcommand{\NormalTok}[1]{#1}
\usepackage{graphicx,grffile}
\makeatletter
\def\maxwidth{\ifdim\Gin@nat@width>\linewidth\linewidth\else\Gin@nat@width\fi}
\def\maxheight{\ifdim\Gin@nat@height>\textheight\textheight\else\Gin@nat@height\fi}
\makeatother
% Scale images if necessary, so that they will not overflow the page
% margins by default, and it is still possible to overwrite the defaults
% using explicit options in \includegraphics[width, height, ...]{}
\setkeys{Gin}{width=\maxwidth,height=\maxheight,keepaspectratio}
\IfFileExists{parskip.sty}{%
\usepackage{parskip}
}{% else
\setlength{\parindent}{0pt}
\setlength{\parskip}{6pt plus 2pt minus 1pt}
}
\setlength{\emergencystretch}{3em}  % prevent overfull lines
\providecommand{\tightlist}{%
  \setlength{\itemsep}{0pt}\setlength{\parskip}{0pt}}
\setcounter{secnumdepth}{0}
% Redefines (sub)paragraphs to behave more like sections
\ifx\paragraph\undefined\else
\let\oldparagraph\paragraph
\renewcommand{\paragraph}[1]{\oldparagraph{#1}\mbox{}}
\fi
\ifx\subparagraph\undefined\else
\let\oldsubparagraph\subparagraph
\renewcommand{\subparagraph}[1]{\oldsubparagraph{#1}\mbox{}}
\fi

%%% Use protect on footnotes to avoid problems with footnotes in titles
\let\rmarkdownfootnote\footnote%
\def\footnote{\protect\rmarkdownfootnote}

%%% Change title format to be more compact
\usepackage{titling}

% Create subtitle command for use in maketitle
\newcommand{\subtitle}[1]{
  \posttitle{
    \begin{center}\large#1\end{center}
    }
}

\setlength{\droptitle}{-2em}
  \title{Similarity}
  \pretitle{\vspace{\droptitle}\centering\huge}
  \posttitle{\par}
  \author{Shantam Gupta}
  \preauthor{\centering\large\emph}
  \postauthor{\par}
  \predate{\centering\large\emph}
  \postdate{\par}
  \date{May 10, 2018}


\begin{document}
\maketitle

\section{Using similarity for Statistical Process Control \&
Monitoring}\label{using-similarity-for-statistical-process-control-monitoring}

Train data represented as S0(reference data) with N0 as the number of
train data points available(entire historical record). Test data
represented as Sw(Data)

\begin{Shaded}
\begin{Highlighting}[]
\NormalTok{Train <-}\StringTok{ }\KeywordTok{read.csv}\NormalTok{(}\StringTok{'lumos_training_set.csv'}\NormalTok{)}
\NormalTok{Test <-}\StringTok{ }\KeywordTok{read.csv}\NormalTok{(}\StringTok{'lumos_all_set.csv'}\NormalTok{)}

\CommentTok{#remove repeated measurements and reshape the dataset}
\NormalTok{ind <-}\StringTok{ }\KeywordTok{which}\NormalTok{(}\KeywordTok{with}\NormalTok{( Train, (Train}\OperatorTok{$}\NormalTok{PepSeq}\OperatorTok{==}\StringTok{"EYEATLEEC(Carbamidomethyl)C(Carbamidomethyl)AK"} \OperatorTok{|}\StringTok{ }\NormalTok{Train}\OperatorTok{$}\NormalTok{PepSeq}\OperatorTok{==}\StringTok{"TC(Carbamidomethyl)VADESHAGC(Carbamidomethyl)EK"}\NormalTok{) ))}
\NormalTok{S0<-Train[}\OperatorTok{-}\NormalTok{ind,]}
\NormalTok{S0<-S0[,}\OperatorTok{-}\DecValTok{2}\NormalTok{]}
\NormalTok{Train<-S0}
\NormalTok{S0}\OperatorTok{$}\NormalTok{PepSeq<-}\StringTok{ }\KeywordTok{gsub}\NormalTok{(}\StringTok{"}\CharTok{\textbackslash{}\textbackslash{}}\StringTok{(Carbamidomethyl}\CharTok{\textbackslash{}\textbackslash{}}\StringTok{)"}\NormalTok{,}\StringTok{""}\NormalTok{,S0}\OperatorTok{$}\NormalTok{PepSeq)}
\NormalTok{S0 <-}\StringTok{ }\KeywordTok{reshape}\NormalTok{(S0, }\DataTypeTok{idvar =} \StringTok{"idfile"}\NormalTok{, }\DataTypeTok{timevar =} \StringTok{"PepSeq"}\NormalTok{, }\DataTypeTok{direction =} \StringTok{"wide"}\NormalTok{)}
\NormalTok{RESPONSE<-}\KeywordTok{c}\NormalTok{(}\StringTok{"GO"}\NormalTok{)}
\NormalTok{S0 <-}\StringTok{ }\KeywordTok{cbind}\NormalTok{(S0,RESPONSE)}

\NormalTok{ind <-}\StringTok{ }\KeywordTok{which}\NormalTok{(}\KeywordTok{with}\NormalTok{( Test, (Test}\OperatorTok{$}\NormalTok{PepSeq}\OperatorTok{==}\StringTok{"EYEATLEEC(Carbamidomethyl)C(Carbamidomethyl)AK"} \OperatorTok{|}\StringTok{ }\NormalTok{Test}\OperatorTok{$}\NormalTok{PepSeq}\OperatorTok{==}\StringTok{"TC(Carbamidomethyl)VADESHAGC(Carbamidomethyl)EK"}\NormalTok{) ))}
\NormalTok{Data0<-Test[}\OperatorTok{-}\NormalTok{ind,]}
\NormalTok{Data0<-Data0[,}\OperatorTok{-}\DecValTok{2}\NormalTok{]}
\NormalTok{Data0}\OperatorTok{$}\NormalTok{PepSeq<-}\StringTok{ }\KeywordTok{gsub}\NormalTok{(}\StringTok{"}\CharTok{\textbackslash{}\textbackslash{}}\StringTok{(Carbamidomethyl}\CharTok{\textbackslash{}\textbackslash{}}\StringTok{)"}\NormalTok{,}\StringTok{""}\NormalTok{,Data0}\OperatorTok{$}\NormalTok{PepSeq)}
\NormalTok{Data1 <-}\StringTok{ }\NormalTok{Data0[}\DecValTok{1}\OperatorTok{:}\DecValTok{8} \OperatorTok{+}\StringTok{ }\KeywordTok{rep}\NormalTok{(}\KeywordTok{seq}\NormalTok{(}\DecValTok{0}\NormalTok{, }\KeywordTok{nrow}\NormalTok{(Data0), }\DataTypeTok{by=}\DecValTok{100}\NormalTok{), }\DataTypeTok{each=}\DecValTok{8}\NormalTok{),]}
\NormalTok{Data1 <-}\StringTok{ }\KeywordTok{reshape}\NormalTok{(Data1, }\DataTypeTok{idvar =} \StringTok{"idfile"}\NormalTok{, }\DataTypeTok{timevar =} \StringTok{"PepSeq"}\NormalTok{, }\DataTypeTok{direction =} \StringTok{"wide"}\NormalTok{)}
\NormalTok{RESPONSE<-}\KeywordTok{c}\NormalTok{(}\StringTok{"NOGO"}\NormalTok{)}
\NormalTok{Data <-}\StringTok{ }\KeywordTok{cbind}\NormalTok{(Data1,RESPONSE)}
\end{Highlighting}
\end{Shaded}

\section{Installing the Package}\label{installing-the-package}

\begin{Shaded}
\begin{Highlighting}[]
\CommentTok{#install.packages("lsa")}
\KeywordTok{library}\NormalTok{(lsa) }\CommentTok{# cosine}
\end{Highlighting}
\end{Shaded}

\begin{verbatim}
## Loading required package: SnowballC
\end{verbatim}

\begin{Shaded}
\begin{Highlighting}[]
\CommentTok{#?cosine}
\end{Highlighting}
\end{Shaded}

\section{Filtering numeric features}\label{filtering-numeric-features}

\begin{Shaded}
\begin{Highlighting}[]
\CommentTok{#makes sure the data is numeric}
\NormalTok{sw <-}\StringTok{ }\KeywordTok{sapply}\NormalTok{(Data[,}\KeywordTok{c}\NormalTok{(}\OperatorTok{-}\DecValTok{1}\NormalTok{,}\OperatorTok{-}\DecValTok{50}\NormalTok{)],as.numeric)}
\NormalTok{s0 <-}\StringTok{ }\KeywordTok{sapply}\NormalTok{(S0[,}\KeywordTok{c}\NormalTok{(}\OperatorTok{-}\DecValTok{1}\NormalTok{,}\OperatorTok{-}\DecValTok{50}\NormalTok{)],as.numeric)}
\end{Highlighting}
\end{Shaded}

\section{Taking colum means to get the average representative point for
train
data(s0)}\label{taking-colum-means-to-get-the-average-representative-point-for-train-datas0}

\begin{Shaded}
\begin{Highlighting}[]
\NormalTok{avg_s0 <-}\StringTok{ }\KeywordTok{colMeans}\NormalTok{(s0)}
\end{Highlighting}
\end{Shaded}

\section{Finding similarity between train and test data
point(Exploration)}\label{finding-similarity-between-train-and-test-data-pointexploration}

\subsection{Cosine Similarity}\label{cosine-similarity}

\subsubsection{Train data}\label{train-data}

\begin{Shaded}
\begin{Highlighting}[]
\NormalTok{train_cs <-}\StringTok{ }\KeywordTok{cosine}\NormalTok{(}\KeywordTok{t}\NormalTok{(}\KeywordTok{as.matrix}\NormalTok{(s0)))}

\CommentTok{#finding the least similar data points in the matrix}
\KeywordTok{which}\NormalTok{(train_cs }\OperatorTok{==}\StringTok{ }\KeywordTok{min}\NormalTok{(train_cs), }\DataTypeTok{arr.ind =} \OtherTok{TRUE}\NormalTok{)}
\end{Highlighting}
\end{Shaded}

\begin{verbatim}
##      row col
## [1,]  66  50
## [2,]  50  66
\end{verbatim}

The success for this method is determined by the fact how similarity
affects the distribution of the data. Let's look at the distribution of
similarity in train data

\begin{Shaded}
\begin{Highlighting}[]
\KeywordTok{library}\NormalTok{(lattice)}

\KeywordTok{levelplot}\NormalTok{(train_cs)}
\end{Highlighting}
\end{Shaded}

\includegraphics{similarity_files/figure-latex/unnamed-chunk-6-1.pdf}

let us remove the above rows which have low similarity

\begin{Shaded}
\begin{Highlighting}[]
\KeywordTok{levelplot}\NormalTok{(train_cs[}\OperatorTok{-}\KeywordTok{c}\NormalTok{(}\DecValTok{66}\NormalTok{,}\DecValTok{50}\NormalTok{),}\OperatorTok{-}\KeywordTok{c}\NormalTok{(}\DecValTok{66}\NormalTok{,}\DecValTok{50}\NormalTok{)])}
\end{Highlighting}
\end{Shaded}

\includegraphics{similarity_files/figure-latex/unnamed-chunk-7-1.pdf}

\section{Comparing the test data point with all reference data and
finding the least
similarity}\label{comparing-the-test-data-point-with-all-reference-data-and-finding-the-least-similarity}

\section{Plotting some of the values}\label{plotting-some-of-the-values}

\begin{Shaded}
\begin{Highlighting}[]
\NormalTok{min_similarity <-}\StringTok{ }\KeywordTok{c}\NormalTok{()}
\NormalTok{test_similarity <-}\StringTok{ }\KeywordTok{data.frame}\NormalTok{()}
\CommentTok{#finding cosine similarity between test data point(sw) and train data(s0)}
\ControlFlowTok{for}\NormalTok{(i }\ControlFlowTok{in} \DecValTok{1}\OperatorTok{:}\KeywordTok{nrow}\NormalTok{(sw))\{}
\NormalTok{  matrix <-}\StringTok{ }\KeywordTok{as.matrix}\NormalTok{(}\KeywordTok{rbind}\NormalTok{(s0,sw[i,]))}
\NormalTok{  train_test_i_cs <-}\StringTok{ }\KeywordTok{cosine}\NormalTok{(}\KeywordTok{t}\NormalTok{(}\KeywordTok{as.matrix}\NormalTok{(}\KeywordTok{rbind}\NormalTok{(s0,sw[i,]))))}

  \CommentTok{#taking the last element from the matrix and finding the lowest similarity}
\NormalTok{  min_similarity[i] <-}\StringTok{ }\KeywordTok{min}\NormalTok{(train_test_i_cs[}\DecValTok{108}\NormalTok{,])}
  
  \CommentTok{#storing the similarity of test data points}
\NormalTok{  test_similarity <-}\StringTok{ }\KeywordTok{rbind}\NormalTok{(test_similarity,train_test_i_cs[}\DecValTok{108}\NormalTok{,])}
\NormalTok{\} }
\end{Highlighting}
\end{Shaded}

\begin{Shaded}
\begin{Highlighting}[]
\KeywordTok{names}\NormalTok{(test_similarity) <-}\StringTok{ }\OtherTok{NULL}
\KeywordTok{row.names}\NormalTok{(test_similarity) <-}\StringTok{ }\OtherTok{NULL}
\KeywordTok{levelplot}\NormalTok{(}\KeywordTok{as.matrix}\NormalTok{(test_similarity[}\DecValTok{1}\OperatorTok{:}\DecValTok{100}\NormalTok{,]))}
\end{Highlighting}
\end{Shaded}

\includegraphics{similarity_files/figure-latex/unnamed-chunk-9-1.pdf}

The 107 columns represent the similarity with train data(s0). The rows
are first 100 test data points. For great results the above matrix
should show colors with low similarity value


\end{document}
